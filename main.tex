\documentclass[12pt, a4paper]{article}

% --- BASIC PACKAGES ---
\usepackage[T1]{fontenc}
\usepackage[utf8]{inputenc}
\usepackage[turkish,english]{babel}
\usepackage{csquotes}
\usepackage{lipsum}

% --- FONT SETTINGS ---
\usepackage{newtxtext}
\usepackage{newtxmath}

% --- PAGE LAYOUT & MARGINS ---
\usepackage[left=3cm, right=2cm, top=3cm, bottom=2cm, headheight=15pt]{geometry}

% --- LINE SPACING & PARAGRAPH ---
\usepackage{setspace}
\onehalfspacing

\usepackage{indentfirst}
\setlength{\parindent}{1cm}
\setlength{\parskip}{0pt}

% --- WIDOW AND ORPHAN CONTROL ---
\widowpenalty=10000
\clubpenalty=10000

% --- LIST SETTINGS ---
\usepackage{enumitem}
\setlist{nosep}

% --- SECTION HEADINGS (Titlesec) ---
\usepackage{titlesec}

\titleformat{\section}
  {\clearpage\centering\bfseries\fontsize{14}{16}\selectfont} 
  {\thesection} 
  {1em} 
  {\MakeUppercase} 
  [\vspace{12pt}] 

\titleformat{\subsection}
  {\bfseries\fontsize{12}{14}\selectfont}
  {\thesubsection}
  {1em}
  {}
  [\vspace{6pt}]

\titleformat{\subsubsection}
  {\bfseries\fontsize{12}{14}\selectfont} 
  {\thesubsubsection}
  {1em}
  {}
  [\vspace{0pt}]

\titlespacing*{\section}{0pt}{0pt}{12pt}
\titlespacing*{\subsection}{0pt}{12pt}{6pt}
\titlespacing*{\subsubsection}{0pt}{12pt}{0pt}

% --- HEADER & FOOTER ---
\usepackage{fancyhdr}
\pagestyle{fancy}
\fancyhf{}
\fancyhead[R]{\thepage}
\fancyfoot[C]{}
\renewcommand{\headrulewidth}{0pt}

% --- FOOTNOTE SETTINGS ---
\usepackage{footmisc}

\renewcommand{\footnoterule}{%
    \vspace{1em}%
    \noindent\rule{5cm}{0.4pt}%
    \vspace{1em}%
}

\renewcommand{\footnotesize}{\fontsize{10}{12}\selectfont}

% --- BIBLIOGRAPHY (IEEE Style) ---
\usepackage[backend=biber, style=ieee, sorting=none]{biblatex}

% IEEE style için volume+number macro düzeltmesi
\renewbibmacro*{volume+number+eid}{%
  \printfield{volume}%
  \setunit*{\addnbspace}%
  \printfield{number}%
  \setunit{\addcomma\space}%
  \printfield{eid}}

\addbibresource{references.bib}

% --- FIGURES & TABLES ---
\usepackage{graphicx}
\usepackage{caption}
\usepackage{float}
\usepackage{booktabs}
\usepackage{multirow}
\usepackage{amsmath}
\usepackage{mathtools}

% --- NUMBERING WITHIN SECTIONS ---
\numberwithin{equation}{section}
\numberwithin{figure}{section}
\numberwithin{table}{section}

% --- CUSTOM FLOAT FOR ALGORITHMS ---
\newfloat{algorithm}{htbp}{loa}
\floatname{algorithm}{Algorithm}
\numberwithin{algorithm}{section}

% Rename captions
\addto\captionsenglish{
  \renewcommand{\contentsname}{\hfill TABLE OF CONTENTS \hfill}
  \renewcommand{\figurename}{Figure}
  \renewcommand{\tablename}{Table}
}

\captionsetup[table]{skip=10pt, position=top}
\captionsetup[figure]{skip=10pt, position=bottom}

% --- TOC SETTINGS ---
\usepackage{tocloft}
\renewcommand{\cftsecleader}{\cftdotfill{\cftdotsep}}
\renewcommand{\cftsecfont}{\normalfont} 
\renewcommand{\cftsecpagefont}{\normalfont}
\renewcommand{\cftdotsep}{1}
\setlength{\cftbeforesecskip}{0pt}


\begin{document}
\shorthandoff{=}

% ==========================================
% TITLE PAGE
% ==========================================
\begin{titlepage}
    \centering
    \vspace*{1cm}
    
    {\bfseries\fontsize{14}{16}\selectfont CMPE351 COMPUTER NETWORKS\\}
    \vspace{12pt}
    {\bfseries\fontsize{14}{16}\selectfont PROJECT REPORT\\}
    \vspace{12pt}
    {\bfseries\fontsize{14}{16}\selectfont Course Project \#1\\}
    
    \vspace{3cm}
    
    {\bfseries\fontsize{16}{18}\selectfont Usage of Ten Different Well-known Ports over TCP and/or UDP in WAN\\}
    
    \vspace{3cm}
    
    \textbf{Submitted by:}\\
    \vspace{12pt}
    Cengiz Demirtaş (CMPE)\\
    İlker Çetin (CMPE)\\
    Mustafa Ünel (CMPE)
    
    \vspace{2cm}
    
    \textbf{Project Supervisor:}\\
    Asst. Prof. Dr. İlktan Ar
    
    \vspace{1cm}
    
    \textbf{Project Mentors:}\\
    Sena Kılınç\\
    Tuna Zeyneloğlu
    
    \vfill
    
    Faculty of Engineering and Natural Sciences\\
    Kadir Has University\\
    November 2025
    
\end{titlepage}

% ==========================================
% FRONT MATTER
% ==========================================
\pagenumbering{arabic} 
\setcounter{page}{2} 

\tableofcontents
\thispagestyle{empty} 
\newpage

% ==========================================
% MAIN CONTENT
% ==========================================

\section{INTRODUCTION}
\label{sec:introduction}
The objective of this project is to analyze the behavior of well-known ports in a Wide Area Network (WAN). Understanding port usage is critical for network security and traffic management. As stated in the project guidelines \cite{source1}, we focus on ten distinct ports.\footnote{This footnote demonstrates the custom footnote formatting with adjusted spacing and rule.}

Network communication relies heavily on the Transport Layer protocols, specifically TCP and UDP. The fundamental theory of communication was established years ago \cite{shannon1948}, but modern applications require more complex handling as discussed in recent studies \cite{lecun2015}.

\section{METHODOLOGY}
\label{sec:methodology}

\subsection{Experimental Setup}
We set up a client-server architecture to test port accessibility. The detailed topology is illustrated in Figure \ref{fig:topology}. We used packet sniffing tools to capture the traffic data.

\begin{figure}[H]
    \centering
    \includegraphics[width=0.6\textwidth]{example-image}
    \caption{Network topology used for port scanning}
    \label{fig:topology}
\end{figure}

As seen in Figure \ref{fig:topology}, the client connects through a firewall. The configuration ensures that only specific packets are allowed.

\subsection{Mathematical Model}
To calculate the round-trip time (RTT), we used the standard formula. Equation \ref{eq:rtt} shows the calculation for the estimated RTT:

\begin{equation}
    RTT_{new} = (1 - \alpha) \cdot RTT_{old} + \alpha \cdot RTT_{sample}
    \label{eq:rtt}
\end{equation}

where $\alpha$ is the weighting factor. We referenced Equation \ref{eq:rtt} throughout our analysis to determine network latency.

\begin{equation}
    \text{Throughput} = \frac{\text{Data Size}}{\text{Transfer Time}}
    \label{eq:throughput}
\end{equation}

\subsection{List Examples}
The following items were considered in our experimental design:

\begin{itemize}
    \item Network topology configuration
    \item Firewall rule definitions
    \item Packet capture methodology
    \item Data analysis procedures
\end{itemize}

Enumerated steps for the experiment:

\begin{enumerate}
    \item Configure the test environment
    \item Deploy monitoring tools
    \item Execute port scanning procedures
    \item Analyze collected data
    \item Generate final report
\end{enumerate}

\section{MAIN FINDINGS}
\label{sec:findings}

\subsection{Port Analysis Results}
The results of our scan are summarized below. We compared TCP and UDP performance across different services.

\begin{table}[htbp]
    \centering
    \caption{Status of well-known ports during the WAN scan}
    \label{tab:ports}
    \begin{tabular}{llcl}
    \toprule
    \textbf{Port} & \textbf{Service} & \textbf{Protocol} & \textbf{Status} \\ 
    \midrule
    20 & FTP Data   & TCP & Closed \\
    21 & FTP Control& TCP & Open \\
    22 & SSH        & TCP & Open \\
    53 & DNS        & UDP & Filtered \\
    80 & HTTP       & TCP & Open \\
    443& HTTPS      & TCP & Open \\
    \bottomrule
    \end{tabular}
\end{table}

Table \ref{tab:ports} clearly indicates that secure ports (like 22 and 443) were left open for administration and web services, while legacy ports like 20 were closed.

\subsubsection{Detailed Comparison}
Comparing the data from Table \ref{tab:ports} with the theoretical values in \cite{shannon1948}, we observe a significant deviation in UDP packet loss.

\subsection{Algorithm Example}
The scanning algorithm used in this project is presented below:

\begin{algorithm}[H]
\caption{Port Scanning Algorithm}
\label{alg:portscan}
\begin{verbatim}
FOR each port in target_port_list DO
    SEND TCP SYN packet to target_host:port
    WAIT for response with timeout
    IF SYN-ACK received THEN
        Mark port as OPEN
    ELSE IF RST received THEN
        Mark port as CLOSED
    ELSE
        Mark port as FILTERED
    END IF
END FOR
\end{verbatim}
\end{algorithm}

Algorithm \ref{alg:portscan} demonstrates the basic port scanning methodology employed in Section \ref{sec:methodology}.

\section{CONCLUSIONS}
\label{sec:conclusions}
In this project, we successfully analyzed ten well-known ports. The results show that firewall rules significantly impact the visibility of UDP ports compared to TCP. Future work should focus on IPv6 compatibility.

% ==========================================
% REFERENCES
% ==========================================
\newpage
\printbibliography[heading=bibintoc, title={REFERENCES}]

% ==========================================
% APPENDICES
% ==========================================
\newpage
\appendix

\addtocontents{toc}{\protect\setlength{\cftsecnumwidth}{8em}} 

\renewcommand{\thesection}{APPENDIX \Alph{section}:} 

\section{ADDITIONAL DATA}
\label{app:data}
This appendix contains raw data logs captured during the experiment.

\begin{table}[H]
    \centering
    \caption{Raw packet capture statistics}
    \label{tab:rawdata}
    \begin{tabular}{lcc}
    \toprule
    \textbf{Metric} & \textbf{TCP} & \textbf{UDP} \\ 
    \midrule
    Total Packets & 15,432 & 8,921 \\
    Dropped Packets & 23 & 156 \\
    Average Latency (ms) & 45.2 & 52.7 \\
    \bottomrule
    \end{tabular}
\end{table}

\section{CONFIGURATION FILES}
\label{app:config}
This appendix includes sample configuration files used during the experiment.

\end{document}
