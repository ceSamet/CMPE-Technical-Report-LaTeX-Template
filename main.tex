\documentclass[12pt, a4paper]{article}

% --- BASIC PACKAGES ---
\usepackage[T1]{fontenc}
\usepackage[utf8]{inputenc}
\usepackage[turkish,english]{babel} % Language support
\usepackage{csquotes} % Recommended for biblatex
\usepackage{lipsum}   % Generates dummy text (lorem ipsum)

% --- FONT SETTINGS ---
% Using 'newtx' for Times New Roman style (text and math)
\usepackage{newtxtext}
\usepackage{newtxmath}

% --- PAGE LAYOUT & MARGINS ---
% Template rules: Left: 3cm, Right: 2cm, Top: 3cm, Bottom: 2cm
\usepackage[left=3cm, right=2cm, top=3cm, bottom=2cm, headheight=15pt]{geometry}

% --- LINE SPACING & PARAGRAPH ---
\usepackage{setspace}
\onehalfspacing % 1.5 line spacing (Template rule)

\usepackage{indentfirst} % Indent the first paragraph of sections
\setlength{\parindent}{1cm} % 1cm indentation
\setlength{\parskip}{0pt}   % No extra space between paragraphs

% --- SECTION HEADINGS (Titlesec) ---
\usepackage{titlesec}

% 1. Section (Level 1): 14pt, Bold, Centered, Uppercase, New Page
\titleformat{\section}
  {\clearpage\centering\bfseries\fontsize{14}{16}\selectfont} 
  {\thesection} 
  {1em} 
  {\MakeUppercase} 
  [\vspace{12pt}] 

% 2. Subsection (Level 2): 12pt, Bold, Left Aligned
\titleformat{\subsection}
  {\bfseries\fontsize{12}{14}\selectfont}
  {\thesubsection}
  {1em}
  {}
  [\vspace{6pt}]

% 3. Subsubsection (Level 3): 12pt, Bold (assumed based on common practice), Left Aligned
\titleformat{\subsubsection}
  {\bfseries\fontsize{12}{14}\selectfont} 
  {\thesubsubsection}
  {1em}
  {}
  [\vspace{0pt}]

% Spacing around headings
\titlespacing*{\section}{0pt}{0pt}{12pt}
\titlespacing*{\subsection}{0pt}{12pt}{6pt}
\titlespacing*{\subsubsection}{0pt}{12pt}{0pt}

% --- HEADER & FOOTER ---
\usepackage{fancyhdr}
\pagestyle{fancy}
\fancyhf{} % Clear all fields
\fancyhead[R]{\thepage} % Page number at top right
\renewcommand{\headrulewidth}{0pt} % No header line

% --- BIBLIOGRAPHY (IEEE Style) ---
\usepackage[backend=biber, style=ieee, sorting=none]{biblatex}
\addbibresource{references.bib} % Pointing to your .bib file

% --- FIGURES & TABLES ---
\usepackage{graphicx} % For images
\usepackage{caption}  % For caption formatting
\usepackage{float}    % For [H] placement
\usepackage{booktabs} % For professional looking tables (toprule, midrule)
\usepackage{multirow} % For merging rows in tables

% Rename captions if using English (Explicitly ensuring formats)
\addto\captionsenglish{
  \renewcommand{\contentsname}{\hfill TABLE OF CONTENTS \hfill}
  \renewcommand{\figurename}{Figure}
  \renewcommand{\tablename}{Table}
}

% Caption positioning and spacing
\captionsetup[table]{skip=10pt, position=top}
\captionsetup[figure]{skip=10pt, position=bottom}

% --- TOC SETTINGS ---
\usepackage{tocloft}
\renewcommand{\cftsecleader}{\cftdotfill{\cftdotsep}} % Dots for sections in TOC
\renewcommand{\cftsecfont}{\normalfont} 
\renewcommand{\cftsecpagefont}{\normalfont}


\begin{document}
\shorthandoff{=}

% ==========================================
% TITLE PAGE
% ==========================================
\begin{titlepage}
    \centering
    \vspace*{1cm}
    
    {\bfseries\fontsize{14}{16}\selectfont CMPE351 COMPUTER NETWORKS\\}
    \vspace{12pt}
    {\bfseries\fontsize{14}{16}\selectfont PROJECT REPORT\\}
    \vspace{12pt}
    {\bfseries\fontsize{14}{16}\selectfont Course Project \#1\\}
    
    \vspace{3cm}
    
    {\bfseries\fontsize{16}{18}\selectfont Usage of Ten Different Well-known Ports over TCP and/or UDP in WAN\\}
    
    \vspace{3cm}
    
    \textbf{Submitted by:}\\
    \vspace{12pt}
    Cengiz Demirtaş (CMPE)\\
    İlker Çetin (CMPE)\\
    Mustafa Ünel (CMPE)
    
    \vspace{2cm}
    
    \textbf{Project Supervisor:}\\
    Asst. Prof. Dr. İlktan Ar
    
    \vspace{1cm}
    
    \textbf{Project Mentors:}\\
    Sena Kılınç\\
    Tuna Zeyneloğlu
    
    \vfill
    
    Faculty of Engineering and Natural Sciences\\
    Kadir Has University\\
    November 2025
    
\end{titlepage}

% ==========================================
% FRONT MATTER
% ==========================================
\pagenumbering{arabic} 
\setcounter{page}{2} 

\tableofcontents
\thispagestyle{empty} 
\newpage

% ==========================================
% MAIN CONTENT
% ==========================================

\section{INTRODUCTION}
The objective of this project is to analyze the behavior of well-known ports in a Wide Area Network (WAN). Understanding port usage is critical for network security and traffic management. As stated in the project guidelines \cite{source1}, we focus on ten distinct ports.

Network communication relies heavily on the Transport Layer protocols, specifically TCP and UDP. The fundamental theory of communication was established years ago \cite{shannon1948}, but modern applications require more complex handling as discussed in recent studies \cite{lecun2015}.

\section{METHODOLOGY}

\subsection{Experimental Setup}
We set up a client-server architecture to test port accessibility. The detailed topology is illustrated in Figure \ref{fig:topology}. We used packet sniffing tools to capture the traffic data.

% Example of a Figure
\begin{figure}[H]
    \centering
    % \includegraphics[width=0.8\textwidth]{topology.png} % UNCOMMENT THIS LINE AND USE YOUR IMAGE
    \includegraphics[width=0.6\textwidth]{example-image} % Placeholder image
    \caption{Network topology used for port scanning}
    \label{fig:topology}
\end{figure}

As seen in Figure \ref{fig:topology}, the client connects through a firewall. The configuration ensures that only specific packets are allowed.

\subsection{Mathematical Model}
To calculate the round-trip time (RTT), we used the standard formula. Equation \ref{eq:rtt} shows the calculation for the estimated RTT:

% Example of an Equation
\begin{equation}
    RTT_{new} = (1 - \alpha) \cdot RTT_{old} + \alpha \cdot RTT_{sample}
    \label{eq:rtt}
\end{equation}

where $\alpha$ is the weighting factor. We referenced Equation \ref{eq:rtt} throughout our analysis to determine network latency.

\section{MAIN FINDINGS}

\subsection{Port Analysis Results}
The results of our scan are summarized below. We compared TCP and UDP performance across different services.

% Example of a Professional Table (using booktabs)
\begin{table}[htbp]
    \centering
    \caption{Status of well-known ports during the WAN scan}
    \label{tab:ports}
    \begin{tabular}{llcl}
    \toprule
    \textbf{Port} & \textbf{Service} & \textbf{Protocol} & \textbf{Status} \\ 
    \midrule
    20 & FTP Data   & TCP & Closed \\
    21 & FTP Control& TCP & Open \\
    22 & SSH        & TCP & Open \\
    53 & DNS        & UDP & Filtered \\
    80 & HTTP       & TCP & Open \\
    443& HTTPS      & TCP & Open \\
    \bottomrule
    \end{tabular}
\end{table}

Table \ref{tab:ports} clearly indicates that secure ports (like 22 and 443) were left open for administration and web services, while legacy ports like 20 were closed.

\subsubsection{Detailed Comparison}
Comparing the data from Table \ref{tab:ports} with the theoretical values in \cite{shannon1948}, we observe a significant deviation in UDP packet loss.

\section{CONCLUSIONS}
In this project, we successfully analyzed ten well-known ports. The results show that firewall rules significantly impact the visibility of UDP ports compared to TCP. Future work should focus on IPv6 compatibility.

% ==========================================
% REFERENCES
% ==========================================
\newpage
% Print bibliography and add it to TOC
\printbibliography[heading=bibintoc, title={REFERENCES}]

% ==========================================
% APPENDICES
% ==========================================
\newpage
\appendix

% --- BU KOMUTU EKLE (ÇAKIŞMAYI BU DÜZELTİR) ---
% İçindekiler tablosundaki numara genişliğini 8 birime çıkarır.
\addtocontents{toc}{\protect\setlength{\cftsecnumwidth}{8em}} 
% -----------------------------------------------

\renewcommand{\thesection}{APPENDIX \Alph{section}:} 

\section{ADDITIONAL DATA}
This appendix contains raw data logs captured during the experiment.

\end{document}
